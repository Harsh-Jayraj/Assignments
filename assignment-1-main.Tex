    \documentclass[a4paper]{article} 
    \addtolength{\hoffset}{-2.25cm}
\addtolength{\textwidth}{4.5cm}
\addtolength{\voffset}{-3.25cm}
\addtolength{\textheight}{5cm}
\setlength{\parskip}{0pt}
\setlength{\parindent}{0in}

%----------------------------------------------------------------------------------------
%	PACKAGES AND OTHER DOCUMENT CONFIGURATIONS
%----------------------------------------------------------------------------------------

\usepackage{blindtext} % Package to generate dummy text
\usepackage{charter} % Use the Charter font
\usepackage[utf8]{inputenc} % Use UTF-8 encoding
\usepackage{microtype} % Slightly tweak font spacing for aesthetics
\usepackage[english, ngerman]{babel} % Language hyphenation and typographical rules
\usepackage{amsthm, amsmath, amssymb} % Mathematical typesetting
\usepackage{float} % Improved interface for floating objects
\usepackage[final, colorlinks = true, 
            linkcolor = black, 
            citecolor = black]{hyperref} % For hyperlinks in the PDF
\usepackage{graphicx, multicol} % Enhanced support for graphics
\usepackage{xcolor} % Driver-independent color extensions
\usepackage{marvosym, wasysym} % More symbols
\usepackage{rotating} % Rotation tools
\usepackage{censor} % Facilities for controlling restricted text
\usepackage{listings, style/lstlisting} % Environment for non-formatted code, !uses style file!
\usepackage{pseudocode} % Environment for specifying algorithms in a natural way
\usepackage{style/avm} % Environment for f-structures, !uses style file!
\usepackage{booktabs} % Enhances quality of tables
\usepackage{tikz-qtree} % Easy tree drawing tool
\tikzset{every tree node/.style={align=center,anchor=north},
         level distance=2cm} % Configuration for q-trees
\usepackage{style/btree} % Configuration for b-trees and b+-trees, !uses style file!
\usepackage[backend=biber,style=numeric,
            sorting=nyt]{biblatex} % Complete reimplementation of bibliographic facilities
\addbibresource{ecl.bib}
\usepackage{csquotes} % Context sensitive quotation facilities
\usepackage[yyyymmdd]{datetime} % Uses YEAR-MONTH-DAY format for dates
\renewcommand{\dateseparator}{-} % Sets dateseparator to '-'
\usepackage{fancyhdr} % Headers and footers
\pagestyle{fancy} % All pages have headers and footers
\fancyhead{}\renewcommand{\headrulewidth}{0pt} % Blank out the default header
\fancyfoot[L]{} % Custom footer text
\fancyfoot[C]{} % Custom footer text
\fancyfoot[R]{\thepage} % Custom footer text
\newcommand{\note}[1]{\marginpar{\scriptsize \textcolor{red}{#1}}} % Enables comments in red on margin

%----------------------------------------------------------------------------------------

    \begin{document}
    
    %-------------------------------
    %	TITLE SECTION
    %-------------------------------
    
    \fancyhead[C]{}
    \hrule \medskip % Upper rule
    \begin{minipage}{0.295\textwidth} 
    \raggedright
    \footnotesize
    Harsh Jayraj\hfill\\   
    19111026\hfill\\
    BIOMED 5th SEM
    \end{minipage}
    \begin{minipage}{0.4\textwidth} 
    \centering 
    \large 
    ASSIGNMENT 1\\
    \normalsize 
    Philosophy of Artificial Intelligence\\ 
    \end{minipage}
    \begin{minipage}{0.295\textwidth} 
    \raggedleft
    \today\hfill\\
    \end{minipage}
    \medskip\hrule 
    \bigskip
    
    %-------------------------------
    %	CONTENTS
    %-------------------------------
    
    \section{Introduction}
    The philosophy of artificial intelligence is a branch of the philosophy of technology that explores artificial intelligence and its implications for knowledge and understanding of intelligence, ethics, consciousness, epistemology, and free will.
    \\
    \\
    The philosophy of artificial intelligence attempts to answer such questions as follows:
    
    Can a machine act intelligently? Can it solve any problem that a person would solve by thinking?\\
    Are human intelligence and machine intelligence the same? Is the human brain essentially a computer?\\
    Can a machine have a mind, mental states, and consciousness in the same sense that a human being can? Can it feel how things are?\\
    \\
    Important propositions in the philosophy of AI include some of the following:\\
    \\
    --Turing's "polite convention": If a machine behaves as intelligently as a human being, then it is as intelligent as a human being.\\
    --The Dartmouth proposal: "Every aspect of learning or any other feature of intelligence can be so precisely described that a machine can be made to simulate it."\\
    --Allen Newell and Herbert A. Simon's physical symbol system hypothesis: "A physical symbol system has the necessary and sufficient means of general intelligent action."\\
    --John Searle's strong AI hypothesis: "The appropriately programmed computer with the right inputs and outputs would thereby have a mind in exactly the same sense human beings have minds."\\
    --Hobbes' mechanism: "For 'reason' ... is nothing but 'reckoning,' that is adding and subtracting, of the consequences of general names agreed upon for the 'marking' and 'signifying' of our thoughts..."\\
    
    \section{Can a Machine display Intelligence?}
    Is it possible for a machine to perform an execute all tasks that a human does?\\
    This Questions arises two main other questions i.e.  whats is Intelligence? and the another that whether a machine lies under the category of being intelligent or not?\\
    \subsection{What is Intelligence?}
    1) according to alan turing - "If a machine acts as intelligently as a human being, then it is as intelligent as a human being."\\
    \\
    2) according to Intelligent Agent Definition - "If an agent acts so as to maximize the expected value of a performance measure based on past experience and knowledge then it is intelligent." where an "agent" is something which perceives and acts in an environment.
    \bigskip
    \subsection{Can display general intelligence or NOT?}
    1) The Brain can be  stimulated.\\
    2) Human thinking is a symbolic processing - In 1963, Allen Newell and Herbert A. Simon proposed that "symbol manipulation" was the essence of both human and machine intelligence.\\
    3) Godelian anti-mechanist arguments  and dreyfus: the primacy of implicit skills pose a huge argument against imtelligence as symbol processing.\\
    4)Today Artificial intelligence have advance so much in fields of computer vision, natural language processing, robotics, machine learning, deep learning.
    %------------------------------------------------
    
    \section{Can Machine Have Mind, Consciousness and a mental state?}
    According to Searle:\\\\
    Strong AI-A physical symbol system can have a mind and mental states.This includes Natural language processing, brain computer interface etc.\\\\
    Weak AI-A physical symbol system can act intelligently. It includes Deep Learning, Machine Learning etc.\\
    \\
    Searle introduced the terms to isolate strong AI from weak AI so he could focus on what he thought was the more interesting and debatable issue. He argued that even if we assume that we had a computer program that acted exactly like a human mind, there would still be a difficult philosophical question that needed to be answered.\\
    \\Neither of Searle's two positions are of great concern to AI research, since they do not directly answer the question "can a machine display general intelligence?"
    \subsection{Consciousness, minds, mental states, meaning}
    The word consciousness according to some philosophers is an invisible, energetic fluid that permeates life and especially the mind.\\\\
    For philosophers, neuroscientists and cognitive scientists, the words are used in a way that is both more precise and more mundane: they refer to the familiar, everyday experience of having a "thought in your head", like a perception, a dream, an intention or a plan, and to the way we know something, or mean something or understand something.
    
    \bigskip
    \subsection{Arguments that a computer cannot have a mind and mental states}
    1) Searle's Chinese room
    2) Leibniz' mill, Davis's telephone exchange, Block's Chinese nation and Blockhead
    \bigskip
    
    \section{Is thinking a kind of computation?}
    The computational theory of mind or "computationalism" claims that the relationship between mind and brain is similar (if not identical) to the relationship between a running program and a computer.\\
    \\if the human brain is a kind of computer then computers can be both intelligent and conscious, answering both the practical and philosophical questions of AI.Some versions of computationalism make the claim that:\\
    \\
    -Reasoning is nothing but reckoning.-  "Our intelligence derives from a form of calculation, similar to arithmetic."\\
    -Mental states are just implementations of (the right) computer programs.\\
    
    \bigskip
    
    \section{Other related Questions on which philosophers argued in this field}
    \subsection{Can a machine have emotions?}
    \subsection{Can a machine be self-aware?}
    \subsection{Can a machine be original or creative?}
    \subsection{Can a machine be benevolent or hostile?}
    \subsection{Can a machine imitate all human characteristics?}
    \subsection{Can a machine have a soul?}
    %------------------------------------------------
    
    \end{document}
