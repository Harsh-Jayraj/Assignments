%%%%%%%%%%%%%%%%%%%%%%%%%%%%%%%%%%%%%%%%%
% Homework Assignment Article
% LaTeX Template
% Version 1.3.5r (2018-02-16)
%
% This template has been downloaded from:
% /cl.uni-heidelberg.de/~zimmermann/
%
% Original author:
% Victor Zimmermann (zimmermann@cl.uni-heidelberg.de)
%
% License:
% CC BY-SA 4.0 (https://creativecommons.org/licenses/by-sa/4.0/)
%
%%%%%%%%%%%%%%%%%%%%%%%%%%%%%%%%%%%%%%%%%

%----------------------------------------------------------------------------------------

\documentclass[a4paper,10pt]{article} % Uses article class in A4 format

%----------------------------------------------------------------------------------------
%	FORMATTING
%----------------------------------------------------------------------------------------

\setlength{\parskip}{0pt}
\setlength{\parindent}{0pt}
\setlength{\voffset}{-15pt}

%----------------------------------------------------------------------------------------
%	PACKAGES AND OTHER DOCUMENT CONFIGURATIONS
%----------------------------------------------------------------------------------------

\usepackage[a4paper, margin=2.5cm]{geometry} % Sets margin to 2.5cm for A4 Paper
\usepackage[onehalfspacing]{setspace} % Sets Spacing to 1.5

\usepackage[T1]{fontenc} % Use European encoding
\usepackage[utf8]{inputenc} % Use UTF-8 encoding
\usepackage{charter} % Use the Charter font
\usepackage{microtype} % Slightly tweak font spacing for aesthetics

\usepackage[english, ngerman]{babel} % Language hyphenation and typographical rules

\usepackage{amsthm, amsmath, amssymb} % Mathematical typesetting
\usepackage{marvosym, wasysym} % More symbols
\usepackage{float} % Improved interface for floating objects
\usepackage[final, colorlinks = true, 
            linkcolor = black, 
            citecolor = black,
            urlcolor = black]{hyperref} % For hyperlinks in the PDF
\usepackage{graphicx, multicol} % Enhanced support for graphics
\usepackage{xcolor} % Driver-independent color extensions
\usepackage{rotating} % Rotation tools
\usepackage{listings, style/lstlisting} % Environment for non-formatted code, !uses style file!
\usepackage{pseudocode} % Environment for specifying algorithms in a natural way
\usepackage{style/avm} % Environment for f-structures, !uses style file!
\usepackage{booktabs} % Enhances quality of tables

\usepackage{tikz-qtree} % Easy tree drawing tool
\tikzset{every tree node/.style={align=center,anchor=north},
         level distance=2cm} % Configuration for q-trees
\usepackage{style/btree} % Configuration for b-trees and b+-trees, !uses style file!

\usepackage{titlesec} % Allows customization of titles
\renewcommand\thesection{\arabic{section}.} % Arabic numerals for the sections
\titleformat{\section}{\large}{\thesection}{1em}{}
\renewcommand\thesubsection{\alph{subsection})} % Alphabetic numerals for subsections
\titleformat{\subsection}{\large}{\thesubsection}{1em}{}
\renewcommand\thesubsubsection{\roman{subsubsection}.} % Roman numbering for subsubsections
\titleformat{\subsubsection}{\large}{\thesubsubsection}{1em}{}

\usepackage[all]{nowidow} % Removes widows

\usepackage[backend=biber,style=numeric,
            sorting=nyt, natbib=true]{biblatex} % Complete reimplementation of bibliographic facilities
\addbibresource{main.bib}
\usepackage{csquotes} % Context sensitive quotation facilities

\usepackage[yyyymmdd]{datetime} % Uses YEAR-MONTH-DAY format for dates
\renewcommand{\dateseparator}{-} % Sets dateseparator to '-'

\usepackage{fancyhdr} % Headers and footers
\pagestyle{fancy} % All pages have headers and footers
\fancyhead{}\renewcommand{\headrulewidth}{0pt} % Blank out the default header
\fancyfoot[L]{\textsc{}} % Custom footer text
\fancyfoot[C]{} % Custom footer text
\fancyfoot[R]{\thepage} % Custom footer text

\newcommand{\note}[1]{\marginpar{\scriptsize \textcolor{red}{#1}}} % Enables comments in red on margin

%----------------------------------------------------------------------------------------

\begin{document}

%----------------------------------------------------------------------------------------
%	TITLE SECTION
%----------------------------------------------------------------------------------------

\title{template_assignment} % Article title
\fancyhead[C]{}
\begin{minipage}{0.295\textwidth} % Left side of title section
\raggedright
Harsh Jayraj\\ % Your lecture or course
\footnotesize % Authors text size
%\hfill\\ % Uncomment if right minipage has more lines
biomedical 5th sem, 19111026 % Your name, your matriculation number
\medskip\hrule
\end{minipage}
\begin{minipage}{0.4\textwidth} % Center of title section
\centering 
\large % Title text size
Assignment 05\\ % Assignment title and number
\normalsize % Subtitle text size
Artificial Intelligence\\ % Assignment subtitle
\end{minipage}
\begin{minipage}{0.295\textwidth} % Right side of title section
\raggedleft
\today\\ % Date
\footnotesize % Email text size
%\hfill\\ % Uncomment if left minipage has more lines
harsh26082001@gmail.com% Your email
\medskip\hrule
\end{minipage}

%----------------------------------------------------------------------------------------
%	ARTICLE CONTENTS
%----------------------------------------------------------------------------------------

% here be dragons
\section{\color{brown}Hidden Markov model}
Hidden Markov Model (HMM) is a statistical Markov model in which the system being modeled is assumed to be a Markov process – call it X – with unobservable ("hidden") states. HMM assumes that there is another process Y whose behavior "depends" on X.

\section{Definition}
Let $X_n$ and $Y_n$ be discrete-time stochastic processes and $n>=1$ . The pair $(X_n,Y_n)$ is a hidden Markov model if

a) $X_n$ is a Markov process whose behavior is not directly observable ("hidden");\\
b) P($Y_n$ $\epsilon $ A|$X_1$ = $x_1$ ......$X_n$ = $x_n$) = P($Y_n$ $\epsilon $ A, $X_n=$ $x_n$)\\
\subsection{Terminology}
The states of the process $X_{n}$ are called hidden states, and  P($Y_n$ $\epsilon $ A, $X_n=$ $x_n$)  is called emission probability or output probability.
\section{Examples}
1) Drawing balls from hidden Urns\\
\\In its discrete form, a hidden Markov process can be visualized as a generalization of the urn problem with replacement (where each item from the urn is returned to the original urn before the next step).[6] Consider this example: in a room that is not visible to an observer there is a genie. The room contains urns X1, X2, X3, ... each of which contains a known mix of balls, each ball labeled y1, y2, y3, ... . The genie chooses an urn in that room and randomly draws a ball from that urn. It then puts the ball onto a conveyor belt, where the observer can observe the sequence of the balls but not the sequence of urns from which they were drawn. The genie has some procedure to choose urns; the choice of the urn for the n-th ball depends only upon a random number and the choice of the urn for the (n − 1)-th ball. The choice of urn does not directly depend on the urns chosen before this single previous urn; therefore, this is called a Markov process.\\\\

2) Weather guessing game\\
\\Consider two friends, Alice and Bob, who live far apart from each other and who talk together daily over the telephone about what they did that day. Bob is only interested in three activities: walking in the park, shopping, and cleaning his apartment. The choice of what to do is determined exclusively by the weather on a given day. Alice has no definite information about the weather, but she knows general trends. Based on what Bob tells her he did each day, Alice tries to guess what the weather must have been like.

Alice believes that the weather operates as a discrete Markov chain. There are two states, "Rainy" and "Sunny", but she cannot observe them directly, that is, they are hidden from her. On each day, there is a certain chance that Bob will perform one of the following activities, depending on the weather: "walk", "shop", or "clean". Since Bob tells Alice about his activities, those are the observations. The entire system is that of a hidden Markov model (HMM).

\section{Structural architecture}
The hidden state space is assumed to consist of one of N possible values, modelled as a categorical distribution. (See the section below on extensions for other possibilities.) This means that for each of the N possible states that a hidden variable at time t can be in, there is a transition probability from this state to each of the N possible states of the hidden variable at time $T+1$, for a total of $N^{2}$ transition probabilities. Note that the set of transition probabilities for transitions from any given state must sum to 1. Thus, the N\times N matrix of transition probabilities is a Markov matrix. Because any one transition probability can be determined once the others are known, there are a total of  N \times (N-1) transition parameters.

In addition, for each of the N possible states, there is a set of emission probabilities governing the distribution of the observed variable at a particular time given the state of the hidden variable at that time. The size of this set depends on the nature of the observed variable. For example, if the observed variable is discrete with M possible values, governed by a categorical distribution, there will be (M-1) separate parameters, for a total of N \times (M-1) emission parameters over all hidden states. On the other hand, if the observed variable is an M-dimensional vector distributed according to an arbitrary multivariate Gaussian distribution, there will be M parameters controlling the means and {\frac {M(M+1)}{2}} parameters controlling the covariance matrix, for a total of 



\math{N\left(M+{\frac {M(M+1)}{2}}\right)={\frac {NM(M+3)}{2}}=O(NM{2})}  emission parameters.

\section{Applications}
Computational finance\\
Single-molecule kinetic analysis\\
Cryptanalysis\\
Speech recognition, including Siri\\
Speech synthesis\\
Part-of-speech tagging\\
Document separation in scanning solutions\\
Machine translation\\
Partial discharge\\
Gene prediction\\
Handwriting recognition\\
Alignment of bio-sequences\\
Time series analysis\\
Activity recognition\\
Protein folding\\
Sequence classification\\
Metamorphic virus detection\\
DNA motif discovery\\
DNA hybridization kinetics\\
Chromatin state discovery\\
Transportation forecasting\\
Solar irradiance variability \\

\end{document}
